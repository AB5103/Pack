\documentclass[A4document,12pt]{article}\usepackage[]{graphicx}\usepackage[]{color}
\makeatletter
\def\maxwidth{ 
  \ifdim\Gin@nat@width>\linewidth
    \linewidth
  \else
    \Gin@nat@width
  \fi
}
\makeatother

\definecolor{fgcolor}{rgb}{0.345, 0.345, 0.345}
\newcommand{\hlnum}[1]{\textcolor[rgb]{0.686,0.059,0.569}{#1}}%
\newcommand{\hlstr}[1]{\textcolor[rgb]{0.192,0.494,0.8}{#1}}%
\newcommand{\hlcom}[1]{\textcolor[rgb]{0.678,0.584,0.686}{\textit{#1}}}%
\newcommand{\hlopt}[1]{\textcolor[rgb]{0,0,0}{#1}}%
\newcommand{\hlstd}[1]{\textcolor[rgb]{0.345,0.345,0.345}{#1}}%
\newcommand{\hlkwa}[1]{\textcolor[rgb]{0.161,0.373,0.58}{\textbf{#1}}}%
\newcommand{\hlkwb}[1]{\textcolor[rgb]{0.69,0.353,0.396}{#1}}%
\newcommand{\hlkwc}[1]{\textcolor[rgb]{0.333,0.667,0.333}{#1}}%
\newcommand{\hlkwd}[1]{\textcolor[rgb]{0.737,0.353,0.396}{\textbf{#1}}}%
\let\hlipl\hlkwb

\usepackage{framed}
\makeatletter
\newenvironment{kframe}{%
 \def\at@end@of@kframe{}%
 \ifinner\ifhmode%
  \def\at@end@of@kframe{\end{minipage}}%
  \begin{minipage}{\columnwidth}%
 \fi\fi%
 \def\FrameCommand##1{\hskip\@totalleftmargin \hskip-\fboxsep
 \colorbox{shadecolor}{##1}\hskip-\fboxsep
     \hskip-\linewidth \hskip-\@totalleftmargin \hskip\columnwidth}%
 \MakeFramed {\advance\hsize-\width
   \@totalleftmargin\z@ \linewidth\hsize
   \@setminipage}}%
 {\par\unskip\endMakeFramed%
 \at@end@of@kframe}
\makeatother

\definecolor{shadecolor}{rgb}{.97, .97, .97}
\definecolor{messagecolor}{rgb}{0, 0, 0}
\definecolor{warningcolor}{rgb}{1, 0, 1}
\definecolor{errorcolor}{rgb}{1, 0, 0}
\newenvironment{knitrout}{}{} % an empty environment to be redefined in TeX

\usepackage{alltt}
\renewcommand{\contentsname}{Inhalt}
\usepackage {placeins}
\usepackage{titlesec}
\usepackage{amsfonts}
\usepackage{amsmath}
\usepackage{mathabx}
\usepackage{amssymb}
\usepackage{rotating}
\usepackage[]{graphicx}
\usepackage[]{color}
\titlelabel{\thetitle.\quad}

\usepackage{changepage}
\renewcommand{\baselinestretch}{1.5} 
\usepackage[margin=1.5 in]{geometry}
\IfFileExists{upquote.sty}{\usepackage{upquote}}{}
\begin{document}
\title{\begin{center}\textbf{Description of the help functions for estimation of the parametrical survival model} \end{center}}
\date{\today}
\maketitle
\small
\pagenumbering{arabic}
\newpage
\section{Preliminaries}
To start computations execute the following commands:
\begin{itemize}
\item Define working directory WD="PathToWorkingDirectory";
\item Put script "HelpFunctionsParamAnalysisNew7.R";
\item Set working directory using operator "setwd(WD)";
\item Install and open packages "xtable", "ucminf","MASS", "survival";
\item execute command "source("HelpFunctionsParamAnalysisNew7.R")";
\item execute command "data(lung)".
\end{itemize}
\noindent
{\textbf{Structure of the data}}
\\
\noindent
The data set includes the following fields:
\begin{itemize}
\item   Time-to-failure and censoring in the case without left truncation or time-of-start , time-of-failure, and censoring in the case with left truncation at the time of begin. Censoring must be either 0 (no event) or 1 (event);\
\item   Covariates (continuous or categorical) used in a study (can be empty set).
\end{itemize}

\clearpage
\section{Description of the help functions}
\noindent
{\textbf{Function $NamFact$.}}\\
\noindent
\textbf{Description}\\
This function returns the list of the factor names in the study after converting categorical variables in binary ones.\\
\textbf{Usage}\\
$NamFact(data,formula.scale,formula.shape)$;\\
\textbf{Arguments}
\\
\noindent
$data$:       The  data set;\\
$formula.scale$:       The  formula object defining the fields for time-to-failure (or time-of-start and time-to-failure) and for covariates influencing the proportional hazard term;\\
\noindent
$formula.shape$:           The object defining the fields for covariates influencing the shape.\\
\textbf{Value}\\
List of the factor names.\\
\noindent
{\textbf{Function $msurv$.}}\\
\noindent
\textbf{Description}\\
This function calculates the marginal survival.\\
\textbf{Usage}\\
$msurv(ID)$;\\
\textbf{Arguments}
\\
\noindent
$ID$: Number of the object in the data set.\\
\textbf{Value}\\
Marginal survival for the object number $ID$.\\
{\textbf{Function $cumhazard$.}}\\
\noindent
\textbf{Description}\\
This function calculates cumulative survival.\\
\textbf{Usage}\\
$cumhazard(ID)$;\\
\textbf{Arguments}
\\
\noindent
$ID$: Number of the object in the data set.\\
\textbf{Value}\\
Cumulative hazard for the object number $ID$.\\
{\textbf{Function $hazard$.}}\\
\noindent
\textbf{Description}\\
This function calculates instant hazard.\\
\textbf{Usage}\\
$hazard(ID)$;\\
\textbf{Arguments}
\\
\noindent
$ID$: Number of the object in the data set.\\
\textbf{Value}\\
Instant hazard for the object number $ID$.\\


\noindent
{\textbf{Function $LikGenNPH$.}}
\\\\
\noindent
\textbf{Description}\\
This function calculates the neglikelihood for parametrical survival model.\\
\textbf{Usage}\\
$LikGenNPH(\theta,D,nf,nk,ncl,dist)$;\\
\textbf{Arguments}
\\
\noindent
$\theta$:  Vector of parameters in the form 
\[\theta=(\log a,\log b,\beta _{shape},\beta _{scale},\log \sigma ^2),\]
where $a$ and $b$ are slope and shape parameters defining the hazard functions (Weibull or Gompertz, see details below), $\beta _{shape}$ and $\beta _{scale}$ are the Cox-regression parameters for shape and scale, respectively, and $\sigma ^2$ is the variance of frailty. This vector must include at least two parameters ($\log a$ and $\log b$); 
\\
\noindent
$D$:  The data set as described in previous section;\\
\noindent
$nf$:       The number of continuous and binary factors in the data set $D$ corresponding to the covariates used in the Cox-regression for proportional hazard term;\\
\noindent
$nk$:        The number of continuous and binary factors in the data set $D$ corresponding to the covariates used in the Cox-regression for shape $b$;\\
\noindent
$ncl$:       The number of clusters in the data set $D$ corresponding to the covariate defining the shared frailty (is equal to 0 for the fixed-effect model);\\
\noindent
$dist$:   distribution of the time-to-failure ('Weibull' or 'Gompertz').\\
\textbf{Value}\\
Negloglikelihood.\\

\noindent
{\textbf{Function $GrGenNPH$.}}
\\\\
\noindent
\textbf{Description}\\
This function calculates the neggradient of the liglikelihood for parametrical survival model.\\
\textbf{Usage}\\
$GrGenNPH(\theta,D,nf,nk,ncl,dist)$;\\
\textbf{Arguments}
\\
\noindent
$\theta$:  Vector of parameters in the form 
\[\theta=(\log a,\log b,\beta _{shape},\beta _{scale},\log \sigma ^2),\]
where $a$ and $b$ are slope and shape parameters defining the hazard functions (Weibull or Gompertz, see details below), $\beta _{shape}$ and $\beta _{scale}$ are the Cox-regression parameters for shape and scale, respectively, and $\sigma ^2$ is the variance of frailty. This vector must include at least two parameters ($\log a$ and $\log b$); 
\\
\noindent
$D$:  The data set as described in previous section;\\
\noindent
$nf$:       The number of continuous and binary factors in the data set $D$ corresponding to the covariates used in the Cox-regression for proportional hazard term;\\
\noindent
$nk$:        The number of continuous and binary factors in the data set $D$ corresponding to the covariates used in the Cox-regression for shape $b$;\\
\noindent
$ncl$:       The number of clusters in the data set $D$ corresponding to the covariate defining the shared frailty (is equal to 0 for the fixed-effect model);\\
\noindent
$dist$:   distribution of the time-to-failure ('Weibull' or 'Gompertz').\\
\textbf{Value}\\
Neggradient of the loglikelihood.\\


\noindent
{\textbf{Function $ParNPHCox$.}}
\\\\
\noindent
\textbf{Description}\\
This function calculates the estimates of unknown parameters, their standard errors, and other attributes.\\
\textbf{Usage}\\
$c(par,se,LogLik,Tab,Names,Conc,pval,p.contrast,pstrata):=\\ParNPHCox(formula.scale,formula.shape,cluster,dist,data,expr,strata)$;\\
\textbf{Arguments}
\\
\noindent
$formula.scale$:       The  formula object defining the fields for time-to-failure (or time-of-start and time-to-failure) and for covariates influencing the proportional hazard term;\\
\noindent
$formula.shape$:           The object defining the fields for covariates influencing the shape;\\
\noindent
$cluster$:   The name of the covariate defining the random effect (is equal to NULL for the fixed-effect model);\\
\noindent
$dist$:              distribution of the time-to-failure ('Weibull' or 'Gompertz');\\
\noindent
$Data$:  Data set.\\
\noindent
$expr$:  vector of expressions for contrasts. NULL, otherwise;\\
\noindent
$strata$:  Full list of the factors used in the study and their levels in the form: \\{\it {strata=list(factor1=level of the factor1,...,last factor=level of the last factor)}}.\\ NULL, otherwise.\\
\textbf{Value}\\
\noindent
$par$:        parameter estimates for vector $(\log a,\log b,\beta _{shape},\beta _{scale},\log \sigma ^2)$;\\
\noindent
$se$:        standard errors for vector $par$;\\
\noindent
$LogLik$:        the value of the loglikelihood;\\
\noindent
$Tabs$:        the table of parameter estimates, their standard errors, and $p$-values in the Latex format;\\
\noindent
$Names$:        names of estimated parameters.\\
\noindent
$Conc$:        concordance and its standard error.\\
\noindent
$pval$:        vector of $p$-values for parameter estimates. For null hypothesis the values of $a,b,\sigma^2$ and Cox-regression coefficients are equal to zero.\\
\noindent
$p.contrast$:        data frame for means, CIs, and $p$-values for contrasts;\\
\noindent
$pstrata$:        data frame for times, means, and 95\% CIs of the marginal survivals, cumulative hazards and hazards. The column names are: \\
"Time", "Mean.survival", "Low.survival", "Upper.survival", "Mean.cumulative.hazard", "Low.cumulative.hazard", "Upper.cumulative.hazard", "Mean.hazard", "Low.hazard", "Upper.hazard".   \\
\noindent
\textbf{Details}\\
\noindent
Two kinds of the time-to-failure distribution are used in this function:
\begin{itemize}
\item Weibull with baseline cumulative hazard function
\[H_{base}(t;a,b)=(t/a)^b;\]
\item Gompertz with baseline cumulative hazard function
\[H_{base}(t;a,b)=(a/b)(\exp(bt)-1).\]
\end{itemize}
\noindent
The cumulative hazard function is defined by
\[H(t;a,b,\beta _{shape},\beta _{scale},\mathbf u)=e^{\beta _{scale}\mathbf u}H_{base}(t;a,be^{\beta _{shape}\mathbf u})\]
\noindent
for the covariate vector $\mathbf u$. Here $e^{\beta _{scale}\mathbf u}$ is the Cox-regression term for proportional hazard and $e^{\beta _{shape}\mathbf u}$ is the Cox-regression term for shape.\\
The univariate survival function is defined by
\begin{eqnarray}
\begin{aligned}
S(t;a,b,\beta _{shape},\beta _{scale},\sigma ^2,\mathbf u)=&\mathbb{E}\exp (-ZH(t;a,b,\beta _{shape},\beta _{scale},\mathbf u))\nonumber \\
=&(1+\sigma ^2H(t;a,b,\beta _{shape},\beta _{scale},\mathbf u))^{-1/\sigma ^2}\nonumber 
\end{aligned}
\end{eqnarray}
for gamma-distributed frailty $Z$ with mean 1 and variance $\sigma ^2$.\\
The 'formula.scale' and 'formula.shape' are formula objects used in the R-package survival and have the form $Surv(time,Cens) \sim factor_1+...+factor_k$ or\\$Surv(start,stop,Cens) \sim factor_1+...+factor_k$, where $factor_i$, $i=1,...,k$, is the name of factor used in the Cox-like regression. Interactions between factors is allowed. If no factores are used in the Cox regression the value of 1 stands on the right-hand of the formula. The records with NA values for factors used in both formulas are excluded from the analysis.\\\\
\noindent
{\it{Remark 1.}} The concordance is evaluated using the function 'survConcordance' from the R-package 'survival'.\\
\noindent
{\it{Remark 2.}} The mean contrasts, their CIs, and their $p$-values are calculated on the basis of the empirical distribution for contrasts constructed using $10^6$ generations of the vector of parameters. It is assumed that this vector is normally distributed with known mean and covariance matrix calculated by parameter estimation. \\
\noindent
{\it{Remark 3.}} In some cases (flat likelihood function, multimodality, etc.) the hessian cannot be correctly estimated and the function issues the following error message: \\
{\it {Error in ParNPHCox(formula.scale, formula.shape, cluster, dist, data = ...) : \\
  hessian cannot be correctly calculated. \\
  Change the model and try again.}} \\
\noindent
It can occur if, for example, the data set is not large enough and the shape and scale parameters compete in searching the maximum likelihood function.  In this case it is recommended to simplify the model excluding some factors from the formulas for shape or scale Cox-regression and try to calculate the estimates again.\\\\
\noindent
{\it{Remark 4.}} The number of factors can increase after conversion of any categorical factor in binary one. For example, if factor "type" has three levels "A", "B", and "C" we will get after conversion two binary factors - "typeB" and "typeC" (factor "typeA" is the baseline one and does not appear in the list of the binary factors). The command "strata=list(...)" must include at most one non-zero level for each non-baseline level of the converted categorical factor. For example, it is correct to write "strata=list(typeB=1,typeC=0,...)" or briefly  "strata=list(typeB=1,...)"  and it is not correct to write "strata=list(typeB=1,typeC=1,...)".\\\\
\noindent
{\it{Remark 5.}} If the covariate is a numerical one but we want to consider it as a categorical one we can convert it in categorical variable using the function "as.factor(variable)".
\newpage
\noindent
\textbf{Example 1}\\
library(ucminf)\\
library(MASS)\\
library(xtable)\\
library(survival)\\
data("lung",package="survival")\\
lung\$sex=lung\$sex-1\\
lung\$status=lung\$status-1\\
lung\$ph.ecog=as.factor(lung\$ph.ecog)\\
formula.scale=as.formula('Surv(time, status) $\sim $ age + sex*ph.ecog')\\
formula.shape=as.formula('Surv(time, status) $\sim $ 1')\\
cluster='ph.karno'\\
dist='Weibull'\\
expr=expression(a -500* b, log(msurv(19)/msurv(14))-1, age.scale,sex.scale)\\
NamFact(lung,formula.scale,formula.shape)\\
$[1]$ "age"   "ph.ecog1"     "ph.ecog2"     "ph.ecog3"     "sex"          "sex:ph.ecog1" "sex:ph.ecog2" "sex:ph.ecog3"\\
strata=list(age=0,sex=1,ph.ecog1=1)\\
c(par,se,LogLik,Tab,Names,Conc,pval,p.contrast,pstrata):=\\
     ParNPHCox(formula.scale,formula.shape,cluster,dist,data=lung,expr,strata)\\

\begin{table}[ht]
\centering
\begin{tabular}{rlll}
  \hline
 & Estimates & CI & p-value \\ 
  \hline
Sample size & 226 &  &  \\ 
  Number of non-censored & 163 &  &  \\ 
  a   & 719.069 & 291.56 - 1771.594 & 0 \\ 
  b & 1.372 & 1.215 - 1.549 & 0 \\ 
  exp(age.scale) & 1.009 & 0.99 - 1.028 & 0.342 \\ 
  exp(sex.scale) & 0.524 & 0.247 - 1.111 & 0.0922 \\ 
  exp(ph.ecog1.scale) & 1.457 & 0.921 - 2.304 & 0.1075 \\ 
  exp(ph.ecog2.scale) & 2.282 & 1.342 - 3.881 & 0.0024 \\ 
  exp(ph.ecog3.scale) & 6.298 & 0.843 - 47.295 & 0.0728 \\ 
  exp(sex:ph.ecog1.scale) & 1.097 & 0.456 - 2.641 & 0.8363 \\ 
  exp(sex:ph.ecog2.scale) & 1.249 & 0.463 - 3.373 & 0.6605 \\ 
  exp(sex:ph.ecog3.scale) & 1 & 1 - 1 & 0.9983 \\ 
  Sigma2 & 0 & 0 - 0 & 0 \\ 
  Concordance (se) & 0.846 (0.026) &  &  \\ 
  Loglik & -1126.33 &  &  \\ 
  AIC & 2274.65 &  &  \\ 
   \hline
\end{tabular}
\caption{Parameter estimates. Weibull model.} 
\end{table}
\begin{table}[ht]
\centering
\begin{tabular}{rlll}
  \hline
 & contrast & CI & p-value \\ 
  \hline
a - 500 * b & 111.8866 & -410.8308-1091.1863 & 0.92 \\ 
  log(msurv(19)/msurv(14)) - 1 & 0.6556 & 0.2013-1.2754 & 0.0016 \\ 
  age.scale & 0.0091 & -0.0097-0.028 & 0.342 \\ 
  sex.scale & -0.6464 & -1.3992-0.1054 & 0.0922 \\ 
  exp(age.scale) & 1.0092 & 0.9903-1.0284 & 0 \\ 
   \hline
\end{tabular}
\caption{Table of contrasts. Weibull model.} 
\end{table}
\clearpage
\newpage
\noindent
\textbf{Example 2}\\
formula.scale=as.formula('Surv(time, status)  $\sim$ sex')\\
formula.shape=as.formula('Surv(time, status)  $\sim$ sex')\\
cluster='pat.karno'\\
dist='Gompertz'\\
expr=expression(a - b, msurv(19) - msurv(14), sex.scale + sex.shape)\\
NamFact(lung,formula.scale,formula.shape)\\
$[1]$ "sex"\\
strata=list(sex=1)\\
c(par,se,LogLik,Tab,Names,Conc,pval,p.contrast,pstrata):=\\
    ParNPHCox(formula.scale,formula.shape,cluster,dist,data=lung,expr,strata)

\begin{table}[ht]
\centering
\begin{tabular}{rlll}
  \hline
 & Estimates & CI & p-value \\ 
  \hline
Sample size & 225 &  &  \\ 
  Number of non-censored & 162 &  &  \\ 
  1000a  & 2.353 & 1.609 - 3.444 & 0 \\ 
  100b & 0.112 & 0.052 - 0.242 & 0 \\ 
  exp(sex.shape) & 2.199 & 0.886 - 5.468 & 0.089 \\ 
  exp(sex.scale) & 0.411 & 0.232 - 0.726 & 0.0021 \\ 
  $\sigma^2$ & 0.062 & 0.006 - 0.594 & 0 \\ 
  Concordance (se) & 0.9 (0.026) &  &  \\ 
  Loglik & -1128.5 &  &  \\ 
  AIC & 2267 &  &  \\ 
   \hline
\end{tabular}
\caption{Parameter estimates. Gompertz model.} 
\end{table}
\begin{table}[ht]
\centering
\begin{tabular}{rlll}
  \hline
 & contrast & CI & p-value \\ 
  \hline
a - b & 0.0012 & -6e-04-0.0027 & 0.1585 \\ 
  msurv(19) - msurv(14) & 0.097 & 0.0475-0.1531 & 2e-04 \\ 
  sex.scale + sex.shape & -0.1016 & -0.7357-0.5336 & 0.7548 \\ 
   \hline
\end{tabular}
\caption{Table of contrasts. Gompertz model.} 
\end{table}
\end{document}

